% Calibrating the panels
\section{Conclusions}

The behavior of the amplitude spectrum obtained from several \gls{crt} modules under different bias voltages is studied and a method to determine a \gls{sipm}'s gain is developped.
The linear dependency of the gain on the bias voltage is been studied and used to elaborate a calibration method for the \gls{crt} modules.
The improvement of the calibrated amplitude spectrum is significant.

The calibration of the gain is a tidious task, with many influences of many parameters.
With the help of automated tools, human interaction can be reduced, making the calibration of the gain an affordable task.

The selection rules for the validity of the peaks in the fitting procedure can be improved to achieve smaller uncertainties.
Including the second and third most frequent distance between the peaks to reduce the gain's uncertainty needs to be studied.

The additional studies indicate an important dependency of the spectrum on the power supply voltage.
This effects need to be taken into consideration during and after \gls{crt} module calibration.

The study of the calibration results for varying power supply voltages is recommended.

The computed gains and bias settings of the calibration of the \gls{crt} module 75 is found in table \ref{tab:results}.


\begin{table*}
  \centering
  \begin{tabular}{ l c c c }
    Aimed gains &  75 & 85 & 65 \\
      & \begin{tabular}{c c c}
          \gls{sipm} & bias & gain \\
          \toprule
		0 & 192 & 77.5 (7.4) \\
		1 & 195 & 81.8 (2.6) \\
		2 & 192 & 81.3 (8.3) \\
		3 & 194 & 79.4 (5.2) \\
		4 & 196 & 80.5 (5.4) \\
		5 & 188 & 77.9 (7.0) \\
		6 & 189 & 79.6 (5.1) \\
		7 & 205 & 81.3 (6.4) \\
		8 & 207 & 77.8 (3.4) \\
		9 & 201 & 80.8 (5.1) \\
		10 & 209 & 79.8 (3.8) \\
		11 & 206 & 79.6 (3.1) \\
		12 & 207 & 78.8 (6.3) \\
		13 & 203 & 78.6 (3.0) \\
		15 & 215 & 76.4 (2.2) \\
		18 & 215 & 81.1 (3.6) \\
		19 & 193 & 78.9 (3.6) \\
		20 & 199 & 80.9 (3.2) \\
		21 & 192 & 79.6 (4.1) \\
		22 & 195 & 78.9 (4.8) \\
		23 & 202 & 81.0 (5.6) \\
		24 & 189 & 79.8 (5.5) \\
		25 & 210 & 81.0 (6.5) \\
		26 & 230 & 81.6 (3.9) \\
		27 & 198 & 79.3 (3.6) \\
		30 & 217 & 78.2 (4.0) \\
          \bottomrule
        \\
        \end{tabular}
      & \begin{tabular}{c c c}
          \gls{sipm} & bias & gain \\
          \toprule
		0 & 209 & 87.9 (5.7) \\
		1 & 219 & 91.8 (6.2) \\
		2 & 211 & 89.8 (5.9) \\
		3 & 212 & 88.3 (4.5) \\
		4 & 221 & 92.8 (5.4) \\
		5 & 212 & 88.3 (5.7) \\
		6 & 208 & 87.9 (5.5) \\
		7 & 233 & 95.0 (5.2) \\
		8 & 227 & 86.4 (3.6) \\
		9 & 222 & 88.9 (5.8) \\
		10 & 231 & 89.3 (3.6) \\
		11 & 225 & 87.1 (7.3) \\
		12 & 226 & 88.7 (4.4) \\
		13 & 223 & 88.4 (7.4) \\
		15 & 234 & 85.8 (2.5) \\
		18 & 237 & 88.4 (3.7) \\
		19 & 211 & 88.0 (3.2) \\
		20 & 214 & 88.1 (3.6) \\
		21 & 210 & 88.3 (4.7) \\
		22 & 214 & 85.5 (4.2) \\
		23 & 218 & 87.5 (7.9) \\
		24 & 212 & 89.7 (4.4) \\
		25 & 230 & 89.6 (4.1) \\
		26 & 249 & 89.8 (4.4) \\
		27 & 214 & 87.8 (6.6) \\
		30 & 235 & 87.6 (2.2) \\
          \bottomrule
        \\
        \end{tabular}
      & \begin{tabular}{c c c}
          \gls{sipm} & bias & gain \\
          \toprule
		0 & 174 & 70.6 (3.2) \\
		1 & 172 & 67.5 (7.3) \\
		2 & 173 & 68.2 (6.2) \\
		3 & 176 & 69.8 (5.7) \\
		4 & 171 & 65.6 (7.1) \\
		5 & 164 & 65.8 (6.4) \\
		6 & 169 & 65.0 (12.1) \\
		7 & 177 & 67.8 (12.4) \\
		8 & 188 & 68.8 (2.1) \\
		9 & 181 & 72.1 (6.4) \\
		10 & 186 & 64.3 (8.0) \\
		11 & 188 & 71.1 (3.6) \\
		12 & 188 & 69.4 (3.3) \\
		13 & 183 & 69.5 (2.6) \\
		15 & 196 & 67.6 (2.6) \\
		18 & 192 & 69.1 (4.2) \\
		19 & 175 & 69.8 (3.2) \\
		20 & 184 & 73.0 (2.8) \\
		21 & 173 & 68.5 (8.8) \\
		22 & 176 & 67.0 (6.9) \\
		23 & 186 & 71.4 (4.7) \\
		24 & 167 & 65.0 (10.7) \\
		25 & 190 & 71.1 (2.5) \\
		26 & 212 & 72.4 (8.3) \\
		27 & 181 & 71.6 (7.3) \\
		30 & 198 & 67.0 (5.5) \\
          \bottomrule
        \\
        \end{tabular}
    \\
  \end{tabular}
  \\
  \caption{%
    Results of the calibration of \gls{crt} module 75 for a power supply voltage of 5.5 (0.1) V.
    The resulting settings for the bias changing \gls{dac} and corresponding computed gains are listed for the three aimed gains.
    The gain is given in \gls{adc} counts per photo-electron.
  }
  \label{tab:results}
\end{table*}

