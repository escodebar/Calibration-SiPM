%% Theory
\chapter{Neutrino oscillations}

\section{The theoretical model}
\cite{Wang:2015rma}
Assuming neutrinos of the flavors $\nu_\alpha \in \{\nu_e, \nu_\mu, \nu_\tau\}$ and neutrinos of the masses $\nu_i \in \{\nu_1, \nu_2, \nu_3\}$, the relation between a flavor neutrino and the mass neutrinos is given then by the following equation:

\begin{equation}
    \nu_\alpha = \mathbb{U}_{\alpha i} \nu_i
\end{equation}

The \emph{mixing matrix} $\mathbb{U}_{\alpha i}$ is also called \emph{PMNS-matrix}\marginnote{\ldots or simply \emph{Pontecorvo-Maki-Nakagawa-Sakata matrix}} and it is a 3 dimensional, unitary matrix.
The states are given by the following relations:

\begin{align}
    \ket{\nu_\alpha} &= \mathbb{U}^*_{\alpha i} \ket{\nu_i} \\
    \ket{\nu_i}      &= \mathbb{U}_{\alpha i} \ket{\nu_\alpha}
\end{align}

Following the rules of quantum mechanics the probability of a neutrino oscillating from flavor $\alpha$ to flavor $\beta$ is given by the square of the modulus of the \emph{matrix element}:
\begin{align}
	P_{\alpha \to \beta}	&= |\bra{\nu_\beta} \mathbb{P} \ket{\nu_\alpha}|^2 \\
	&= |\bra{\nu_j} \mathbb{U}^*_{\beta j} \delta_{i j} \mathbb{U}_{\alpha i} \ket{\nu_i}|^2
\end{align}

For simplicity, let's assume the \emph{propagation operator} to be $\mathbb{P} = e^{i \phi}$, where $\phi$ is just a phase.


\begin{equation}
    P_{e\alpha} = \delta_{e\alpha} - (2\delta_{e\alpha} - 1) \sin^2 (2\theta) \sin^2 (1.27 \frac{\Delta m^2 L}{E})
    \label{eq:oscillations}
\end{equation}

Let's derive now formula \ref{eq:oscillations} for oscillations in two dimensions, i.e. assuming two different flavors only:
\begin{align}
    \nonumber
    \bra{\nu_e} &= \cos(\theta)\bra{\nu_1} + \sin(\theta)\bra{\nu_2}\\
    \nonumber
    \ket{\nu_e} &= \cos(\theta)\ket{\nu_1} + \sin(\theta)\ket{\nu_2}
\end{align}

Applying the time evolution operator one gets
\begin{align}
    \nonumber
    \ket{\nu_\mu (t)} &= \cos(\theta) \exp(-iE_1t) \ket{\nu_1} + \sin(\theta) \exp(-iE_2t) \ket{\nu_2}
\end{align}

out of which the probability ca be computed:
\begin{align}
    \nonumber
    \braket{\nu_\mu | \nu_\mu(t)} &= & \left( \cos(\theta) \bra{\nu_1} + \sin(\theta) \bra{\nu_2} \right) \\
    \nonumber
    & &\left(\cos(\theta) e^{iE_1t} \ket{\nu_1} + \sin(\theta) e^{iE_2t} \ket{\nu_2} \right)
\end{align}

\subsection{Neutrino masses}
\lipsum[21]
\cite{Gerbino:2015ixa}
\lipsum[22]

\subsection{Mixing angles}
\lipsum[23]
\cite{Guo:2007ug}
\lipsum[24]

\newpage

