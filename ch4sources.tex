% The short baseline neutrino detector
\section{Neutrino sources}

Neutrinos have a very small cross-section and interact very rarely\marginnote{\ldots and matter is virtually transparent for them}.
Many neutrinos emerge from natural and non natural sources adding an important signal background to any observation.
It is important to quantify the background's value of any source whose neutrinos may interact in our detector and bias our observations.
This section covers known natural and human induced neutrino sources.

\subsection{Natural sources}

\paragraph{Geoneutrinos} are neutrino emitted in a $\beta$-decay of a radionuclide naturally occurring in the Earth.
Most geoneutrinos are electron antineutrinos and originate from $\beta^-$-decay-branches of $\ce{^{40}K}$, $\ce{^{232}Th}$ and $\ce{^{238}U}$.
Referr to Geo-neutrinos\cite{2013PrPNP:73:1B} for a detailed review and analysis of the results from the KamLAND and Borexino data.

\paragraph{Atmospheric neutrinos} result from the interaction of cosmic rays with an atomic nucleus in the Earth's atmosphere.
These interactions generate showers of unstable particles -- mostly pions ($\pi$) -- whose decay involves the production of neutrinos.
For more information on cosmic rays refer to section on background detection and mitigation..
The atmospheric neutrino flux is studied using the data of the IceCube experiment\cite[-6em]{Aartsen:2016xlq} and Super-Kamiokande\cite[-1em]{Richard:2015aua}.

\paragraph{Solar neutrinos} The greatest neutrino background contribution is made by the sun.
The main solar neutrino radiation\marginnote{86\% of the solar neutrinos are produced by: $p + p \to d + e^+ + \nu_e$} comes from the proton-proton reaction, which is one of the known fusion reactions by which stars convert hydrogen to helium.
Important contributions are made by the reactions of beryllium and boron\marginnote[-.5em]{%
\setlength{\parindent}{0pt}%
$\ce{^7Be} + e^- \to \ce{^7Li} + \nu_e$, \\
$\ce{^8B} \to \ce{^8Be^*} + e^+ + \nu_e$}.
Solar neutrinos are best reviewed with the data of the Sudbury Neutrino Observatory\cite{Bellerive:2016byv}.

\subsection{Other sources}

\paragraph{Reactor neutrinos} Neutrinos emerging from interactions in nuclear reactors\marginnote{Nuclear scientific reactors, powerplants, nuclear submarines, etc.} are called reactor neutrinos.
The emission spectrum depends strongly on the type of nuclear reactions produced in the reactor\marginnote{So studying a reactor's neutrino spectrum allow to study nuclear power plants.}.

\paragraph{Neutrino beams} When particles in rest decay by a two body decay, the energy and momenta of the resulting particles are known exactly.
It is well known that the lightest charged mesons ($\pi^\pm$) decay into muons and their associated neutrinos, therefore the previous effect can be taken in advantage to build a neutrino beam out of decaying charged pions.

